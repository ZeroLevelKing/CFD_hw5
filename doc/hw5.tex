\documentclass[UTF8]{ctexart}
\usepackage{listings}
\usepackage{booktabs}  
\usepackage{geometry}  
\usepackage{hyperref}
\usepackage{graphicx} 
\usepackage{xcolor}
\usepackage{float}
\usepackage{array}
\usepackage{enumitem}
\usepackage{amsmath,amssymb}
\graphicspath{{figure/}} % 指定放置图片的子文件夹路径
\geometry{a4paper, left=2.5cm, right=2.5cm, top=2.5cm, bottom=2.5cm}

\definecolor{codegreen}{rgb}{0,0.6,0}
\definecolor{codegray}{rgb}{0.5,0.5,0.5}
\definecolor{codepurple}{rgb}{0.58,0,0.82}

\lstset{
    basicstyle=\ttfamily\footnotesize,
    breaklines=true,
    frame=single,
    numbers=left,
    numberstyle=\tiny\color{codegray},
    keywordstyle=\color{blue},
    commentstyle=\color{codegreen},
    stringstyle=\color{codepurple},
    showstringspaces=false
}


\begin{document}

\title{计算流体力学第五次作业}
\author{朱林-2200011028}
\date{\today}
\maketitle


\section{数理算法原理}




\section{代码生成与调试}


\section{结果讨论和物理解释}



%附录
\newpage
\appendix
\section{AI工具使用声明表}
\begin{table}[H]
    \centering
    \begin{tabular}{c|c|c}
        \hline
        使用内容 & 使用比例 & 使用目的 \\ \hline
        hw4.tex & 60\% & 调整pdf格式,调用宏包,省略插入图片和代码的重复性工作 \\ 
        .gitignore & 100\% & 针对于python和latex的.gitignore文件,完全由Copilot生成  \\
        ReadMe & 80\% & 介绍文件,从上次作业继承,结合AI修改 \\
        common.py & 30\% & 主要迭代和划分网格自己实现,部分绘图代码AI生成  \\
        task1.py & 0\% & 自己实现 \\
        task2.py & 0\% & 自己实现 \\
        task3.py & 0\% & 自己实现 \\
        \hline
    \end{tabular}
    \label{tab:AI_tools}
\end{table}
\end{document}